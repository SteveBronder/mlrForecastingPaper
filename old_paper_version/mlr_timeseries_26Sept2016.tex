\documentclass[12pt]{article}%[final]
\setcounter{secnumdepth}{3}
% if you need to pass options to natbib, use, e.g.:
% \PassOptionsToPackage{numbers, compress}{natbib}
% before loading nips_2016
%
% to avoid loading the natbib package, add option nonatbib:
% \usepackage[nonatbib]{nips_2016}

%\usepackage{nips_2016}

% to compile a camera-ready version, add the [final] option, e.g.:
%\usepackage[final]{nips_2016}
\usepackage{amsmath}
\usepackage[utf8]{inputenc} % allow utf-8 input
\usepackage[T1]{fontenc}    % use 8-bit T1 fonts
\usepackage{hyperref}       % hyperlinks
\hypersetup{
     colorlinks   = true,
     citecolor    = black,
     linkcolor    = black
}
\usepackage{url}            % simple URL typesetting
\usepackage{booktabs}       % professional-quality tables
\usepackage{amsfonts}       % blackboard math symbols
\usepackage{nicefrac}       % compact symbols for 1/2, etc.
\usepackage{microtype}      % microtypography
\usepackage{float}
\usepackage{amssymb}
\usepackage{amsthm}
\usepackage{color,soul}
\usepackage{amsmath}
\usepackage{algorithm}
\usepackage[noend]{algpseudocode}
\usepackage[margin=1.5in]{geometry}
\usepackage[english]{babel}
\usepackage{graphicx}
\usepackage{longtable}
\usepackage{verbatim}
\usepackage{setspace}
\usepackage{subcaption}
\doublespacing

\newtheorem{theorem}{Theorem}[section]
\newtheorem{lemma}[theorem]{Lemma}
\newtheorem{proposition}[theorem]{Proposition}
\newtheorem{corollary}[theorem]{Corollary}
\theoremstyle{definition}
\newtheorem{definition}{Definition}[section]

% Code defs from jss
\newcommand\code{\@codex}
\def\@codex#1{{\normalfont\ttfamily\hyphenchar\font=-1 #1}}
%%\let\code=\texttt
\let\proglang=\textsf
\newcommand{\pkg}[1]{{\fontseries{b}\selectfont #1}}


%\theoremstyle{definition}
%\newtheorem{definition}{Definition}[section]


\makeatletter
\def\BState{\State\hskip-\ALG@thistlm}
\makeatother

\title{Time Series Methods in the R package \pkg{mlr}}

% The \author macro works with any number of authors. There are two
% commands used to separate the names and addresses of multiple
% authors: \And and \AND.
%
% Using \And between authors leaves it to LaTeX to determine where to
% break the lines. Using \AND forces a line break at that point. So,
% if LaTeX puts 3 of 4 authors names on the first line, and the last
% on the second line, try using \AND instead of \And before the third
% author name.
\singlespacing
\author{
  Steve Bronder \\
  %Quantitative Methods of the Social Sciences\\
  %Columbia University\\
  %New York City, NY 10027 \\
  \texttt{sab2287@columbia.edu} \\
  %% examples of more authors test etst
   %Department of Computer Science \\
   %Columbia University\\
   %New York City, NY 10027 \\
  %% \AND
  %% Coauthor \\
  %% Affiliation \\
  %% Address \\
  %% \texttt{email} \\
  %% \And
  %% Coauthor \\
  %% Affiliation \\
  %% Address \\
  %% \texttt{email} \\
  %% \And
  %% Coauthor \\
  %% Affiliation \\
  %% Address \\
  %% \texttt{email} \\
}
\doublespacing

\begin{document}
% \nipsfinalcopy is no longer used

\maketitle

\begin{abstract}
The \pkg{mlr} package is a unified interface for machine learning tasks such as classification, regression, cluster analysis, and survival analysis. \pkg{mlr} handles the data pipeline of preprocessing, resampling, model selection, model tuning, ensembling, and prediction. This paper details new methods for developing time series models in \pkg{mlr}. This extension includes standard and novel tools such as Lambert $W\times F_X$ transform data generating processes, autoregressive data generating schemes for forecasting with machine learning models, fixed and growing window cross-validation, and forecasting models in the context of univariate and multivariate time series. Examples are given to demonstrate the benefits of a unified framework for machine learning and time series.
\end{abstract}


\section{Introduction}
There has been a rapid development in time series methods over the last 25 years~\cite{Hyndman25} whereby time series models have not only become more common, but more complex. The \proglang{R} language~\cite{Rbase} has several task views that list the extensive amount of packages available for forecasting, time series methods, and empirical finance. While the amount of packages is large, the open source nature of \proglang{R} has left users without a standard syntactic framework whereby individual packages will have a sub-culture of style, syntax, and output. This causes users unnecessary burden, forcing them to memorize and interpret each authors unique style. The \pkg{mlr}~\cite{mlr} package, short for Machine Learning in R, is a meta-package which works to give a strong syntactic framework for the modeling pipeline. By automating and standardizing the tools and methodologies in machine learning, \pkg{mlr} reduces error from the user during the modeling process. 

This extension to \pkg{mlr} is the first \proglang{R} package, to this authors knowledge, which gives a fully standardized framework for rigorously testing and training forecasting models\footnote{It is important to recognize that packages like \pkg{forecast} and \pkg{BigVAR} have some pieces of the training framework such as windowing cross-validation and tuning of some of the hyperparameters}. While there are some time series methods available in \pkg{caret}~\cite{caret}, development of forecasting models in \pkg{caret} is difficult due to computational constraints and design choices within the package. The highly modular structure of \pkg{mlr} makes it the best option for implementing time series methods and models. This paper will show how using \pkg{mlr}'s strong syntactic structure allows for time series packages such as \pkg{forecast}~\cite{HyndForecast}, \pkg{rugarch}~\cite{rugarch}, and \pkg{BigVAR}~\cite{BigVAR} to use machine learning methodologies such as automated parameter tuning, data preprocessing, model blending, cross-validation, performance evaluation, and parallel processing techniques for increasing model predictive power.

Section~\ref{sec:m4data} describes the time series data used in this paper to showcase how forecasting is performed in \pkg{mlr}. Section~\ref{sec:task} covers how to create and use univariate and multivariate forecasting tasks. Creating, training, updating, and predicting with basic univariate and multivariate forecasting learners are exampled in section~\ref{seq:build} while tuning and resampling strategies for forecasting are in section~\ref{sec:resamp}. Section~\ref{sec:preproc} covers how using $AR(p,d)$ data generating processes allow for machine learning models to be used in the context of forecasting. Lambert $W\times F_X$ data preprocessing is covered in section~\ref{sec:lambert}. Section~\ref{sec:stackfore} demonstrates how univariate and multivariate forecasting learners can be stacked and ensembled with machine learning models.

\section{Forecasting Data}
\label{sec:m4data}

\subsection{Univariate Forecasting Data}
The standard objective in forecasting is, at time period $t$, make predictions of the target variable $y$ for $t+h$ periods into the future. The difference between standard regression tasks and forecasting is that future values of $y$ are mainly predicted by it’s past. This means that forecasting tasks are most suitable when aspects of past patterns will continue on into the future. 

Professional forecasters attempt to predict the future of a series based on its past values. Forecasting has applications in a wide range of tasks including forecasting stock prices~\cite{GRANGER19923}, weather patterns~\cite{MurphymeteoForecast}, international conflicts~\cite{Chadefaux01012014}, and earthquakes~\cite{earthquakeYegu}. This paper uses data from the Makridakis competition To evaluate \pkg{mlr}'s forecasting framework as it is well known and provides several types of time series.

The Makridakis competition~\cite{Makridakis2000451} is a set of forecasting challenges organized by the International Institute of Forecasters and led by Spyros Makridakis to evaluate and compare the accuracy of forecasting methods. The most recent of the competitions, the M4 competition, contains 10,000 time series on a yearly, quarterly, monthly, and daily frequency in areas such as finance, macroeconomics, climate, microeconomics, and industry. A particular climate series is used to example \pkg{mlr}'s forecasting features. The data is daily with the training subset starting on September 6th, 2007 and ending on September 5th, 2009 while the testing subset is from September 6th, 2009 to October 10th, 2009 for a total of 640 training periods and 35 test periods to forecast.

\singlespacing












































































































































